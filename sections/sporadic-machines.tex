\section{5-state Sporadic Machines}\label{sec:sporadic}

\begin{figure}[h!]
    \centering

    % First row: 3 images
    \begin{minipage}{\textwidth}
        \centering
        \begin{subfigure}{0.3\textwidth}
            \centering
            \includegraphics[width=\linewidth]{figures/sporadic-machines/sk1.png}
            \caption*{\href{https://bbchallenge.org/1RB1RD_1LC0RC_1RA1LD_0RE0LB_---1RC}{Skelet \#1}}
        \end{subfigure}
        \hfill
        \raisebox{8.5em}[0pt][0pt]{% <-- tweak this value as needed
            \begin{minipage}{0.3\textwidth}
                \centering
                \includegraphics[width=\linewidth]{figures/sporadic-machines/sk10.png}

                \caption*{\href{https://bbchallenge.org/1RB0RA_0LC1RA_1RE1LD_1LC0LD_---0RB}{Skelet \#10}}
                {\small\emph{Double Fibonacci Counter}}
            \end{minipage}
        }
        \hfill
        \begin{subfigure}{0.3\textwidth}
            \centering
            \includegraphics[width=\linewidth]{figures/sporadic-machines/sk17.png}
            \caption*{\href{https://bbchallenge.org/1RB---_0LC1RE_0LD1LC_1RA1LB_0RB0RA}{Skelet \#17}}
        \end{subfigure}
    \end{minipage}

    \vspace{2.5em}

    % Second row: Shift Overflow Counters
    \begin{tikzpicture}
        \node[draw=magenta, thick, rounded corners, inner sep=8pt] (box1) {
            \begin{minipage}{0.95\textwidth}
                \centering
                \textbf{\textcolor{magenta}{Shift Overflow Counters}}\\[0.8em]
                \begin{subfigure}{0.17\textwidth}
                    \centering
                    \includegraphics[width=\linewidth]{figures/sporadic-machines/soc_sk15.png}
                    \caption*{\href{https://bbchallenge.org/1RB---_1RC1LB_1LD1RE_1LB0LD_1RA0RC}{Skelet \#15}}
                \end{subfigure}
                \hfill
                \begin{subfigure}{0.17\textwidth}
                    \centering
                    \includegraphics[width=\linewidth]{figures/sporadic-machines/soc_sk26.png}
                    \caption*{\href{https://bbchallenge.org/1RB1LD_1RC0RB_1LA1RC_1LE0LA_1LC---}{Skelet \#26}}
                \end{subfigure}
                \hfill
                \begin{subfigure}{0.17\textwidth}
                    \centering
                    \includegraphics[width=\linewidth]{figures/sporadic-machines/soc_sk33.png}
                    \caption*{\href{https://bbchallenge.org/1RB1LC_0RC0RB_1LD0LA_1LE---_1LA1RE}{Skelet \#33}}
                \end{subfigure}
                \hfill
                \begin{subfigure}{0.17\textwidth}
                    \centering
                    \includegraphics[width=\linewidth]{figures/sporadic-machines/soc_sk34.png}
                    \caption*{\href{https://bbchallenge.org/1RB1LC_0RC0RB_1LD0LA_1LE---_1LA1RA}{Skelet \#34}}
                \end{subfigure}
                \hfill
                \begin{subfigure}{0.17\textwidth}
                    \centering
                    \includegraphics[width=\linewidth]{figures/sporadic-machines/soc_sk35.png}
                    \caption*{\href{https://bbchallenge.org/1RB1LC_0RC0RB_1LD0LA_1LE---_1LA0LA}{Skelet \#35}}
                \end{subfigure}
            \end{minipage}
        };
    \end{tikzpicture}

    \vspace{1.5em}

    % Third row: Finned Machines
    \begin{tikzpicture}
        \node[draw=magenta, thick, rounded corners, inner sep=8pt] (box2) {
            \begin{minipage}{0.95\textwidth}
                \centering
                \textbf{\textcolor{magenta}{Finned Machines}}\\[0.8em]
                \begin{subfigure}{0.17\textwidth}
                    \centering
                    \includegraphics[width=\linewidth]{figures/sporadic-machines/finned_1.png}
                    \caption*{\href{https://bbchallenge.org/1RB0LE_1RC1RB_1RD1LC_0LE0RB_---1LA}{Finned \#1}}
                \end{subfigure}
                \hfill
                \begin{subfigure}{0.17\textwidth}
                    \centering
                    \includegraphics[width=\linewidth]{figures/sporadic-machines/finned_2.png}
                    \caption*{\href{https://bbchallenge.org/1RB1RA_1RC1LB_0LD0RA_1RA1LE_---0LD}{Finned \#2}}
                \end{subfigure}
                \hfill
                \begin{subfigure}{0.17\textwidth}
                    \centering
                    \includegraphics[width=\linewidth]{figures/sporadic-machines/finned_3.png}
                    \caption*{\href{https://bbchallenge.org/1RB1RE_1LC1RB_0RA0LD_1LB1LD_---0RA}{Finned \#3}}
                \end{subfigure}
                \hfill
                \begin{subfigure}{0.17\textwidth}
                    \centering
                    \includegraphics[width=\linewidth]{figures/sporadic-machines/finned_4.png}
                    \caption*{\href{https://bbchallenge.org/1RB1LA_0LC0RE_---1LD_1RA0LC_1RA1RE}{Finned \#4}}
                \end{subfigure}
                \hfill
                \begin{subfigure}{0.17\textwidth}
                    \centering
                    \includegraphics[width=\linewidth]{figures/sporadic-machines/finned_5.png}
                    \caption*{\href{https://bbchallenge.org/1RB1LA_0LC0RE_---1LD_1LA0LC_1RA1RE}{Finned \#5}}
                \end{subfigure}
            \end{minipage}
        };
    \end{tikzpicture}

    \caption{{\small Family picture of the 5-state Sporadic Machines (20,000-step space-time diagrams) which required individual \Coq nonhalting proofs; machine names in the Figure are clickable URLs giving the TNF-normalised transition table of each machine (see Section~\ref{sec:enum}). All Sporadic Machines were also identified by Skelet \cite{Skelet_bbfind} as holdouts of his \texttt{bbfind} program. For better visibility, diagrams of counters (Skelet \#10 and Shift Overflow Counters) have been represented using a tape of length 200 instead of 400, giving a \textit{zoomed-in} effect.}}
    \label{fig:sporadic}
\end{figure}

Sporadic Machines are 13 nonhalting 5-state Turing machines that were not captured by deciders (Section~\ref{sec:deciders}) and required individual \Coq proofs of nonhalting; their space-time diagrams are given in Figure~\ref{fig:sporadic} where each name is a clickable URL leading to the machine's transition table and space-time diagram. Twelve of these machines, \ie all but ``Skelet \#17'' (see below), were proved nonhalting in \texttt{busycoq} \cite{busycoq}, and then integrated\footnote{For convenience, the relevant parts of \texttt{busycoq} have been added \href{https://github.com/ccz181078/Coq-BB5/tree/main/BusyCoq}{to the root} of \CoqBB. \CoqBB translates \texttt{busycoq} proofs using \href{https://github.com/ccz181078/Coq-BB5/blob/main/CoqBB5/BB5/BusyCoq_Translation.v}{\texttt{BusyCoq\_Translation.v}}.} into \CoqBB. Machine ``Skelet \#17'' was the last 5-state machine to be formally proven nonhalting in Coq, as part of \CoqBB, achieving the proof of $S(5) = \BBtheFifth$ -- a different proof also had been released as a standalone paper, \cite{xu2024skelet17fifthbusy}.

Interestingly all Sporadic Machines had been identified by Georgi Georgiev (also known as ``Skelet'', see Section~\ref{sec:intro:mainresults}) in 2003: either as part of his 43 unsolved machines\footnote{Apart from \cite{Skelet_bbfind_list}, these 43 machines are also listed here: \url{https://bbchallenge.org/skelet}.} which are named after him, \eg ``Skelet \#1'' (see Figure~\ref{fig:sporadic}), or, in the case of what we call ``Finned Machines'', marked by him as ``easily provable by hand'' \cite{Skelet_bbfind_list}. Sporadic Machines can be arranged in three buckets:
\begin{itemize}
    \item \textbf{Finned Machines.} These are five similar machines that hold 3 unary numbers on the tape (and can merge the middle one into its neighbor), vary them while maintaining a linear relation, and in the process ensure any deviation from this linear relation would be detected and cause a halt. These machines were solved by handcrafting nonhalting certificates similar in flavor to WFAR certificates (Section~\ref{sec:WFAR}). The certificates were crafted by Blanchard, translated in \Coq by mei, see \texttt{busycoq} files \texttt{Finned\{1-5\}.v}. An argument of irregularity (Section~\ref{sec:deciders-overview}) was given for machine ``Finned \#3'' \cite{irregularFinned3}. A later-developed irregular extension of RepWL (Section~\ref{sec:RepWL}) has been reported to solve these machines\footnote{\url{https://discuss.bbchallenge.org/t/bb5s-finned-machines-summary/234}}.
    \item \textbf{Shift Overflow Counters.} This family concerns Skelet's machines \href{https://bbchallenge.org/1RB---_1RC1LB_1LD1RE_1LB0LD_1RA0RC}{15}, \href{https://bbchallenge.org/1RB1LD_1RC0RB_1LA1RC_1LE0LA_1LC---}{26}, \href{https://bbchallenge.org/1RB1LC_0RC0RB_1LD0LA_1LE---_1LA1RE}{33}, \href{https://bbchallenge.org/1RB1LC_0RC0RB_1LD0LA_1LE---_1LA1RA}{34} and \href{https://bbchallenge.org/1RB1LC_0RC0RB_1LD0LA_1LE---_1LA0LA}{35}; Figure~\ref{fig:sporadic}. These machines are similar: they implement two independent binary counters, one to the left of the tape and the other to the right. Their behavior can be described by two distinct phases: an orderly ``Counter Phase'' where each counter is simply incremented and a more complex ``Reset Phase'' triggered by one of the counters overflowing. If the counter were to overflow again during a ``Reset Phase'', the machines would halt. Therefore, the proof of nonhalting depends upon demonstrating that the machine maintains a ``Reset Invariant'' throughout the ``Reset Phase'' which does not allow another overflow. These machines were first analysed and described by Ligocki, who provided an informal proof of ``Skelet \#34'' as well as conjectures about the others \cite{ShawnSOC}. They were then proved in \Coq by Yuen and mei as part of \texttt{busycoq}, see files \texttt{Skelet\{15,26,33,34,35\}.v} \cite{busycoq}.
    \item \textbf{\href{https://bbchallenge.org/1RB1RD_1LC0RC_1RA1LD_0RE0LB_---1RC}{Skelet \#1}, \href{https://bbchallenge.org/1RB0RA_0LC1RA_1RE1LD_1LC0LD_---0RB}{Skelet \#10}, and, \href{https://bbchallenge.org/1RB---_0LC1RE_0LD1LC_1RA1LB_0RB0RA}{Skelet \#17}.} These three machines each have unique behaviors which we detail below.
\end{itemize}
\vspace{-1.5em}
\paragraph{\href{https://bbchallenge.org/1RB1RD_1LC0RC_1RA1LD_0RE0LB_---1RC}{Skelet \#1}.} This machine is a Translated Cycler, \ie a machine that eventually repeats the same pattern translated in space (see Section~\ref{sec:loops}), but with enormous parameters: its pre-period (number of steps to wait before the pattern first appears) is about $5.42 \times 10^{51}$ and its period (number of steps taken by the repeated pattern) is $8,468,569,863$. This was discovered by means of accelerated simulation by Kropitz and Ligocki \cite{uniSk1} and thorough analysis by Ligocki \cite{ShawnSkelet1Before, ShawnSkelet1}. The result was confirmed correct after mei formalised it in \Coq as part of \texttt{busycoq}, see file \texttt{Skelet1.v} \cite{busycoq}. The $10^{51}$ pre-period was computed later by Huang \cite{hipparcosSk1}.
\vspace{-0.5em}
\paragraph{\href{https://bbchallenge.org/1RB0RA_0LC1RA_1RE1LD_1LC0LD_---0RB}{Skelet \#10 (Double Fibonacci Counter)}.} This machine implements two independent \textit{base Fibonacci} counters, one to the left of the tape and the other to the right. Counting in base Fibonacci means exploting Zeckendorf's theorem \cite{wiki:Zeckendorf's_theorem}: any natural number can be expressed as a sum of Fibonacci numbers in exactly one way, excluding using numbers immediately adjacent in the Fibonacci sequence, where the Fibonacci sequence is $F = 1,2,3,5,8,13,21,34\dots$ -- each number in the sequence is the sum of the two previous ones. For instance, $17 = 1 + 3 + 13$ and this decomposition would be represented as \texttt{100101} in big-endian binary: the $i^\text{th}$ bit from the right is $1$ if we use $F_i$ in the sum. Each of the two counters of Skelet \#10 enumerate natural numbers in base Fibonacci, using slightly different encodings and the machine halts iff the counters ever get out of sync -- which, does not happen. The machine was analysed independently by Briggs and Ligocki \cite{DanBriggs,ShawnSkelet10} and Ligocki's proof \cite{ShawnSkelet10} was formalised in \Coq by mei as part of \texttt{busycoq}, see file \texttt{Skelet10.v} \cite{busycoq}. Skelet \#10 is the only known 5-state \textit{double} Fibonacci counter, but there are several known \textit{single} Fibonacci counters, such as \tm{1RB0RA_0LC1RA_1LD0LC_1RE1LC_---0RB}, solved by \CoqBB's NGramCPS (Section~\ref{sec:n-gramCPS}).

% Fibonacci notation is a binary notation where the $n^\text{th}$ bit, instead of representing $2^n$ as in standard binary, represents $F_{n+2}$ where $F = 0,1,1,2,3,5,8,13,21,34\dots$ is the Fibonacci sequence (each number is the sum of the previous two). For instance, in big-endian notation, \texttt{100101} represents $F_{5+2} + F_{2+2} + F_{0+2} = 13 + 3 + 1 = 17$.
\vspace{-0.5em}
\paragraph{\href{https://bbchallenge.org/1RB---_0LC1RE_0LD1LC_1RA1LB_0RB0RA}{Skelet \#17}.} \textit{The final boss.} This machine manages a list of integers $n_1, \dots, n_k \in \mathbb{N}$ represented in unary on the tape using encoding: $(\texttt{10})^{n_1} \texttt{1} (\texttt{10})^{n_2} \texttt{1} \dots \texttt{1} (\texttt{10})^{n_k}$. The list can only increase and undergoes a set of complex transformations related to Gray code, and, the machine halts iff $n_1 = n_2 = 0$ and $n_3,\, \dots,\, n_k$ are all even, which, never happens. This description was first drafted by savask \cite{savaskSk17}, proven in a standalone paper by Xu \cite{xu2024skelet17fifthbusy} and, finally, formalised in \Coq by mxdys as part of \CoqBB. Skelet \#17 was the last 5-state machine to be solved.
