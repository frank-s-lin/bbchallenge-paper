\newpage
\section{Turing machines}\label{sec:TMs}

\vspace{-1.4em}
% \begin{figure}[h!]
%     \centering
%     \begin{subfigure}[t]{0.45\textwidth}
%         \centering
%         \renewcommand{\arraystretch}{1.3} % Increase row height
%         \setlength{\tabcolsep}{12pt} % Increase column spacing
%         \vspace{10pt} % Adjust vertical alignment
%         \begin{tabular}{ccc}
%             \toprule
%                     & \textbf{0} & \textbf{1} \\
%             \midrule
%             \stateA & 1R\stateB  & 1L\stateC  \\
%             \stateB & 1R\stateC  & 1R\stateB  \\
%             \stateC & 1R\stateD  & 0L\stateE  \\
%             \stateD & 1L\stateA  & 1L\stateD  \\
%             \stateE & ---        & 0L\stateA  \\
%             \bottomrule
%         \end{tabular}
%         \caption{Transition table of the 5-state 2-symbol \BBfull winner. This machine was discovered by Marxen and Buntrock in 1989 \cite{Marxen_1990}.}
%         \label{table:bb5}

%         \vspace{10pt} % Adjust spacing
%         \includegraphics[width=0.7\linewidth]{figures/space-time-diagrams/bb5_20k.png} % Adjust the path as needed
%         \caption{20,000-step \textit{zoomed out} space-time diagram of the 5-state winner.}\label{fig:bb5-diagram-zoomout}
%     \end{subfigure}
%     \hfill
%     \begin{subfigure}[t]{0.45\textwidth}
%         \centering
%         \vspace{10pt} % Adjust vertical alignment
%         \includegraphics[width=0.7\linewidth]{figures/space-time-diagrams/bb5.pdf}
%         \caption{Space-time diagram of the first 45 steps of the 5-state winner.}
%         \label{fig:bb5-diagram}
%     \end{subfigure}
%     \caption{Transition table and space-time diagrams of the 5-state 2-symbol busy beaver winner, which halts after 47,176,870 steps.
%         \url{https://bbchallenge.org/1RB1LC_1RC1RB_1RD0LE_1LA1LD_---0LA}.}
%     \label{fig:bb5}
% \end{figure}

\begin{figure}[ht]
    \centering
    \renewcommand{\arraystretch}{1.3}
    \setlength{\tabcolsep}{6pt}

    \begin{subfigure}[b]{0.28\textwidth}
        \centering
        \begin{tabular}{ccc}
            \toprule
                    & \textbf{0} & \textbf{1} \\
            \midrule
            \stateA & 1R\stateB  & 1L\stateC  \\
            \stateB & 1R\stateC  & 1R\stateB  \\
            \stateC & 1R\stateD  & 0L\stateE  \\
            \stateD & 1L\stateA  & 1L\stateD  \\
            \stateE & ---        & 0L\stateA  \\
            \bottomrule
        \end{tabular}
        \caption{5-state 2-symbol \BBfull winner. This machine was discovered by Marxen and Buntrock in 1989 \cite{Marxen_1990}.}
        \label{table:bb5}
    \end{subfigure}
    \hfill
    \begin{subfigure}[b]{0.31\textwidth}
        \centering
        \includegraphics[width=0.60\linewidth]{figures/space-time-diagrams/bb5.pdf}
        \caption{45-step space-time diagram of the 5-state winner. Head position is coloured to indicate state, see (a).}
        \label{fig:bb5-diagram}
    \end{subfigure}
    \hfill
    \begin{subfigure}[b]{0.3\textwidth}
        \centering
        \includegraphics[width=0.8\linewidth]{figures/space-time-diagrams/bb5_20k.png}
        \caption{20,000-step space-time diagram of the 5-state winner.}\label{fig:bb5-diagram-zoomout}
    \end{subfigure}

    \caption{Transition table and space-time diagrams of the 5-state 2-symbol \BBfull winner, which halts after 47,176,870 steps. See
        \tm{1RB1LC_1RC1RB_1RD0LE_1LA1LD_---0LA}.}\label{fig:bb5win}
\end{figure}


In this work, $\N = \{0,1,\dots\}$ and $\N^+ = \{1,2,3\dots\}$.

We consider Turing machines that use a single, discrete, bi-infinite tape, \ie the tape can be thought as a function $\tau: \mathbb{Z} \to \alphabet$, where $\alphabet$ is the alphabet of symbols used by the machine. Machine transitions are either undefined (the machine halts if it ever reaches an undefined transition) or given by (i) a symbol of $\alphabet$ to write; (ii) a direction to move (right or left); and (iii) a state to go to. More precisely, the transition table of a Turing machine is a partially defined function $\delta: \states \times \alphabet \partialto \alphabet \times \{\text{L},\text{R}\} \times \states $, with $\states$ the set of states, \eg $\{\stateA,\stateB,\stateC,\stateD,\stateE\}$ for 5-state machines. Figure~\ref{fig:bb5win}(a) gives the transition table of the 5-state 2-symbol \BBfull winner. The machine halts after 47,176,870 steps (starting from all-0 tape) when it reads a \szero in state \stateE (undefined transition). Allowing for undefined transitions is a small, consequenceless but useful (see Section~\ref{sec:enum}) deviation from \rado's original setup.

In the \BBfull context, machines are always executed from the all-0 tape and starting in state~\stateA. Execution goes as follows: at each step, the machine which is in state $s$ looks at which symbol $\sigma$ is present on the tape cell the head is currently on and then, if defined, executes the instruction given by its transition table, \eg $\delta(s, \sigma) = \szero\text{L}\text{\stateE}$ means that the machine will write a \szero, move the head one cell to the left and switch to state \stateE. If $\delta(s, \sigma)$ is not defined, the machine halts.



A \textit{configuration} (also known as \textit{execution state}) of a Turing machine is defined by the 3-tuple: (i) state; (ii) position of the head on the tape; (iii) content of the tape. As mentioned above, here, \textit{the initial configuration} of a machine is always (i) state is A, i.e. the first state to appear in the machine's description; (ii) head's position is 0; (iii) the initial tape is all-0 -- i.e. each tape cell is containing 0. We write $c_1 \TMstep_\mathcal{M} c_2$ if a configuration $c_2$ is obtained from $c_1$ in one computation step of machine $\mathcal{M}$. We omit $\mathcal{M}$ if it is clear from context. We let $c_1 \TMstep^s c_2$ denote a sequence of $s$ computation steps, and let $c_1 \TMstep^* c_2$ denote zero or more computation steps. % exact same wording as in https://dna.hamilton.ie/assets/dw/NearyWoodsFCT09.pdf

\vspace{-1ex}
\paragraph{Halting and step count convention.} Halting happens when the machine attempts to run an undefined transition. We write $c \TMstep \bot$ to signify that the machine halts after attempting to run one step from configuration $c$. The number of steps $s\in \N^+$ run by a halting Turing machine includes the final halting step, \eg $s = 1$ for a machine where $\delta(\stateA,\symbolzero)$ is not defined.

% When discussing concrete configurations, we write
% $0^\infty\; s_1\; \cdots\; s_{k-1}\; [s_k]_q\; s_{k+1}\; \cdots\; s_n\; 0^\infty$
% to mean the configuration where the machine is in state $q$, with the head
% positioned on the symbol $s_k$, and the tape both starts and end by an infinite sequence of 0s, represented $0^\infty$. Thus, the initial configuration of the machine
% can be written as $0^\infty\; [0]_A\; 0^\infty$.

% \paragraph*{Directional Turing machines.} We will sometimes prefer to think of the tape head as being between
% symbols. Thus, we write $l \lhead{q} r$, with $l,r\in\{0,1\}^*$ to mean that the head is at the rightmost symbol
% of $l$, and $l \rhead{q} r$ to mean that the head is at the leftmost symbol of $r$.
% For example, $0^\infty\; [1]_A\; 0^\infty$ can be written as
% $0^\infty\; 1 \lhead A 0^\infty$ or $0^\infty \rhead A 1\; 0^\infty$.

\vspace{-1ex}
\paragraph*{Turing machine format.} We often communicate Turing machines using the following linear format: \\ \verb|1RB1LC_1RC1RB_1RD0LE_1LA1LD_---0LA| represents the transition table of Figure~\ref{fig:bb5win}(a), where \texttt{\_} is used to separate states and transitions are given in read-symbol order. Note that, historically, the undefined transition reached by a halting machine was represented using \texttt{1RZ}, hence our format allows the use of any letter outside of the state space to represent halting, \eg \texttt{1RB1LC\_1RC1RB\_1RD0LE\_1LA1LD\_1RZ0LA}, the use of \texttt{1RZ} instead of \verb|---| means that \textit{we know} that the machine halts. Multi-symbol machines are represented in the same way, \eg the 2-state 4-symbol \BBfull winner is \verb|1RB2LA1RA1RA_1LB1LA3RB---| (also given by \texttt{1RB2LA1RA1RA\_1LB1LA3RB1RZ}); see Figure~\ref{fig:bb2x4}. This format can be used as URL on \url{bbchallenge.org} to display space-time diagrams and known information about the machine, e.g. \url{https://bbchallenge.org/1RB1LC\_1RC1RB\_1RD0LE\_1LA1LD\_---0LA}.

\vspace{-1ex}
\paragraph*{Space-time diagrams.} We use space-time diagrams to give a visual representation of the behaviour of a given machine. The space-time diagram of machine $\mathcal{M}$ is an image where the $i^\text{th}$ row of the image gives:
\begin{enumerate}
    \item The content of the tape after $i$ steps (for 2-symbol machines, black is 0 and white is 1, while for $n$ symbols, black is 0, white is symbol $n-1$ and linear gray-scaling is used in between, \eg~Figure~\ref{fig:bb2x4}).
    \item The position of the head is coloured to give state information using the following colours for 5-state machines: \textcolor{colorA}{A},  \textcolor{colorB}{B},  \textcolor{colorC}{C},  \textcolor{colorD}{D},  \textcolor{colorE}{E} (one has to look at the row above to deduce what symbol the head is reading, unless it is the initial row, where a \szero is read).
\end{enumerate}

Figure~\ref{fig:bb5win}(b) gives a 45-step space-time diagram for the 5-state 2-symbol \BBfull winner. We often use \textit{zoomed-out} space-time diagrams without state-coloring information, such as Figure~\ref{fig:bb5win}(c), which gives the first 20,000 steps of the 5-state 2-symbol \BBfull winner. Zoomed-out space-time diagrams depicted in this work use a tape of 400 cells unless stated otherwise; the initial cell is generally at the center of the tape but sometimes offset to the right or left. Figure~\ref{fig:bb2x4} showcases the 2-state 4-symbol \BBfull winner.


\begin{figure}[ht]
    \centering
    \renewcommand{\arraystretch}{1.3}
    \setlength{\tabcolsep}{6pt}

    \begin{subfigure}[b]{0.28\textwidth}
        \centering
        \scalebox{0.93}{
            \begin{tabular}{ccccc}
                \toprule
                        & \textbf{0}~\tikz\fill[black, draw=black] (0,0) rectangle (0.20,0.20);     & \textbf{1}~\tikz\fill[graySymb1, draw=black] (0,0) rectangle (0.20,0.20);


                        & \textbf{2}~\tikz\fill[graySymb2, draw=black] (0,0) rectangle (0.20,0.20); & \textbf{3}~\tikz\fill[white, draw=black] (0,0) rectangle (0.20,0.20);                             \\
                \midrule
                \stateA & 1R\stateB                                                                 & 2L\stateA                                                                 & 1R\stateA & 1R\stateA \\
                \stateB & 1L\stateB                                                                 & 1L\stateA                                                                 & 3R\stateB & ---       \\
                \bottomrule
            \end{tabular}
        }
        \caption{Transition table of the 2-state 4-symbol \BBfull winner found by Ligocki and Ligocki in 2005 \cite{PMichel_website}.}
        \label{table:bb2x4}
    \end{subfigure}
    \hfill
    \begin{subfigure}[b]{0.31\textwidth}
        \centering
        \includegraphics[width=0.60\linewidth]{figures/space-time-diagrams/bb2x4.pdf}
        \caption{45-step space-time diagram of the 2-state 4-symbol. Head position is coloured to indicate state, see (a).}
        \label{fig:bb2x4-diagram}
    \end{subfigure}
    \hfill
    \begin{subfigure}[b]{0.32\textwidth}
        \centering
        \includegraphics[width=0.8\linewidth]{figures/space-time-diagrams/bb2x4_20k.png}
        \caption{20,000-step space-time diagram of the 2-state 4-symbol winner.}
        \label{fig:bb2x4-diagram-zoomout}
    \end{subfigure}

    \caption{Transition table and space-time diagrams of the 2-state 4-symbol \BBfull winner, which halts after $\BBTxF$ steps. See
        \tm{1RB2LA1RA1RA\_1LB1LA3RB---}.}
    \label{fig:bb2x4}
\end{figure}